\phantomsection\label{sec:raysearch}
\cventry{2019--2021}{Researcher within Data Science}{RaySearch Laboratories}{Stockholm}{}{
    At RaySearch Laboratories, I worked on automating radiotherapy planning using a combination of deep learning, probabilistic modelling, and multi-objective optimisation. I developed a pipeline that extracted latent anatomical features from segmented 3D CT scans using autoencoders, and used these to identify similar patients via a learned similarity metric. Dose-volume histograms and spatial dose distributions were then predicted using sparse Gaussian processes, providing a probabilistic framework for dose planning conditioned on patient geometry. This enabled personalised, data-driven dose estimation and laid the groundwork for more robust automation in treatment.
    \newline\hspace*{2em}
    Alongside this, I redesigned some of RaySearch's lexicographic optimisation algorithms to better reflect clinical decision hierarchies, particularly in palliative care where trade-offs between tumour control and organ sparing are nuanced. I collaborated closely with medical physicists and clinicians throughout, ensuring the models aligned with both oncological standards and practical workflow requirements.
}{}