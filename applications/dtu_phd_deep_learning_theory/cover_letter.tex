% %--------------------------
% %	CONFIGURATION
% %--------------------------
\documentclass[11pt,a4paper]{moderncv}
\moderncvstyle{classic}
\moderncvcolor{black}
\definecolor{firstnamecolor}{rgb}{0,0,0}

% Basic packages
\usepackage[utf8]{inputenc}
\usepackage[T1]{fontenc}
\usepackage[scale=0.7]{geometry}

\usepackage{microtype} % For better typography

% Configure font settings
\renewcommand{\rmdefault}{ppl} % Use Palatino font
\renewcommand{\sfdefault}{phv} % Use Helvetica font for sans-serif text
\renewcommand*{\namefont}{\fontsize{26}{18}\mdseries}

% Configure microtype
\microtypesetup{expansion=false}

% Load sensitive information
\input{../../personal_info.tex}

% %--------------------------

\begin{document}
\makecvtitle
Dear Georgios Arvanitidis, I am writing to apply for the PhD position in Deep Learning Theory and Differential Geometry. With a Master's degree in data science and applied mathematics, and 6 years of professional experience working with machine learning, I am excited to return to academia. I wish to pursue deep, curiosity-driven research to further my learning and your research topic sounds interesting. 

\hspace*{2em}
Directly after my studies I continued work at RaySearch Laboratories in Stockholm with whom I had written my Master's thesis. My experience at RaySearch as a Data Science Researcher left a strong, long-lasting impression. I developed ML-based methods for radiotherapy planning of cancer treatment. The work was challenging and rewarding, and involved building custom autoencoders for computer vision. Though I had to leave Stockholm in 2021 to move to Copenhagen, my motivation to return to more research heavy work has only grown stronger. 

\hspace*{2em}
At RaySearch, and later at Valcon, I worked with autoencoders and probabilistic models to extract structure from complex data like CT scans and financial systems. These projects gave me practical experience with learning useful representations from high-dimensional input, and sparked an interest in how and why such models work. I have since become increasingly curious about the deeper principles behind how machines \textit{learn} and structure in machine learning, and would like to explore these topics more seriously.

\hspace*{2em}
I have always enjoyed thinking about machine learning from a theoretical angle, and my favourite courses at university were the ones that focused on the underlying mathematics. Over time, I have realised that I want to understand these systems more deeply, not just build them. That is why I am now looking to return to academia and focus on research questions that explore what gives models their ability to generalise and what role structure plays in that process.

\hspace*{2em}
Outside of work, I fill my free time with designing furniture, woodworking,  coding, 3D-printing, climbing, road biking, ultimate frisbee, and board games, activities that keep my hands and mind sharp. I thoroughly enjoy picking up new skills and am self taught in all my hobbies. In my professional life, I now want to return to a research environment where I can follow ideas from early exploration to experimentation, refinement, and ultimately publication.

\vspace{8 mm}
I hope to hear back from you soon. Regards, 

\vspace{3 mm} 
Ivar Cashin Eriksson.

\end{document}