% %--------------------------
% %	CONFIGURATION
% %--------------------------
\documentclass[11pt,a4paper]{moderncv}
\moderncvstyle{classic}
\moderncvcolor{black}
\definecolor{firstnamecolor}{rgb}{0,0,0}

% Basic packages
\usepackage[utf8]{inputenc}
\usepackage[T1]{fontenc}
\usepackage[scale=0.7]{geometry}
\usepackage[dvipsnames]{xcolor}
\definecolor{DarkBlue}{RGB}{0, 0, 150} % or tweak as needed
\usepackage[colorlinks=true, urlcolor=DarkBlue, linkcolor=DarkBlue]{hyperref}

\usepackage{microtype} % For better typography

% Configure font settings
\renewcommand{\rmdefault}{ppl} % Use Palatino font
\renewcommand{\sfdefault}{phv} % Use Helvetica font for sans-serif text
\renewcommand*{\namefont}{\fontsize{26}{18}\mdseries}

% Configure microtype
\microtypesetup{expansion=false}

% Load sensitive information
\input{../../personal_info.tex}

% %--------------------------
\begin{document}
\makecvtitle
Hello Morten Mørup, I am writing to apply for the PhD position in Machine Learning using Tensor Networks. With a Master's degree in data science and applied mathematics, and 6 years of professional hands-on experience working with machine learning, I am excited to return to academia. I wish to pursue deep, curiosity-driven research to further my learning, and your research topic sounds like a good fit for me.

\hspace*{2em}
I hold a degree in Engineering Physics from KTH Royal Institute of Technology in Stockholm, a rigorous and mathematically demanding program covering both theoretical and applied mathematics, as well as classical and modern physics. In my master's, I specialised in applied mathematics, earning a degree in Statistical Learning and Data Analytics with a perfect GPA of 5.0/5.0. This track focused heavily on the mathematical foundations of machine learning, including deep learning. I thoroughly enjoyed understanding how ML models work under the hood, and this intellectual curiosity is what initially led me to pursue a career in data science. Now, I am eager to return to academia to deepen my understanding and help advance the field further.

\hspace*{2em}
I began my career at RaySearch Laboratories in Stockholm, where I also completed my master's thesis. As a Data Science Researcher, I developed probabilistic and computer vision-based ML methods for radiotherapy planning, including autoencoders and lexicographic optimisation. I collaborated closely with oncologists to ensure clinical relevance. Since moving to Copenhagen, I've worked as a Data Scientist across industries, designing real-time multi-modal ML systems and mentoring junior colleagues. These roles gave me a practical engineering perspective on end-to-end ML pipelines, an asset in research aimed at real-world impact.

\hspace*{2em}
Outside of work, I fill my free time with designing furniture, woodworking, coding, 3D-printing, climbing, road biking, ultimate frisbee, and board games, activities that keep my hands and mind sharp. I thoroughly enjoy picking up new skills and am self-taught in all my hobbies. In my professional life, I now want to return to a research environment where I can follow ideas from early exploration to experimentation, refinement, and ultimately publication. I find long-term, curiosity-driven projects deeply motivating, especially when they combine theory, technical implementation, and collaboration across disciplines. I am drawn to the possibility of using tensor networks to improve interpretability of deep neural networks. Explainability is often a key consideration in my work, and has been a blocker in deploying ML models in the past.

\vspace{8 mm}
I hope to hear back from you soon. Regards, 

\vspace{3 mm} 
Ivar Cashin Eriksson.

\end{document}
