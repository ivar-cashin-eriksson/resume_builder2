% %--------------------------
% %	CONFIGURATION
% %--------------------------
\documentclass[11pt,a4paper]{moderncv}
\moderncvstyle{classic}
\moderncvcolor{black}
\definecolor{firstnamecolor}{rgb}{0,0,0}

% Basic packages
\usepackage[utf8]{inputenc}
\usepackage[T1]{fontenc}
\usepackage[scale=0.85]{geometry}
\usepackage[dvipsnames]{xcolor}
\definecolor{DarkBlue}{RGB}{0, 0, 150} % or tweak as needed
\usepackage[colorlinks=true, urlcolor=DarkBlue, linkcolor=DarkBlue]{hyperref}

\usepackage{microtype} % For better typography

% Configure font settings
\renewcommand{\rmdefault}{ppl} % Use Palatino font
\renewcommand{\sfdefault}{phv} % Use Helvetica font for sans-serif text
\renewcommand*{\namefont}{\fontsize{26}{18}\mdseries}

% Configure microtype
\microtypesetup{expansion=false}

% Load sensitive information
\input{../../personal_info.tex}

% Title document
\title{Curriculum Vitae}

% %--------------------------

\begin{document}
\makecvtitle

\section*{Introduction}
I am an curious Data Scientist with 6 years of experience applying ML across many industries. I have previous experience as a data science researcher building computer vision models in radiotherapy for cancer treatment. I have a strong mathematical background and perfect GPA. After some time in industry, I am excited to return to academia to further my learning and contribute to AI research.


% Include modular sections

\section{Selected Work Experience}
\large{Data Science Researcher @ RaySearch Laboratories}

\normalsize
\begin{itemize}
    \item Designed \textbf{3D computer-vision} pipeline for automated cancer treatment planning.
    \item Built lexicographic dose-optimisation algorithms for palliative care. Continuously validated use case with clinical partners.
\end{itemize}

\small{\textit{Keywords:} 3D computer vision, volumetric CNN, autoencoders, medical imaging, optimisation, healthcare AI}

\hfill{\small{\textit{\hyperref[sec:raysearch]{Read more…}}}}

% Two blank spaces for break between sections
\large{Principal Data Scientist @ Valcon}

\normalsize
\begin{itemize}  
    \item Led real-time NLP + multi-modal ML pipelines (50 000 req s$^{-1}$) across pharma, IoT \& energy; Docker + Azure end-to-end deployment.
    \item Mentored 8 junior DS/DEs; created workshops on XAI \& Git; won “People's Choice 2023” award for leadership.
\end{itemize}

\small{\textit{Keywords:} Real-Time ML, NLP, Multi-Modal, Docker, Leadership}

\hfill{\small{\textit{\hyperref[sec:signum]{Read more…}}}}

% Two blank spaces for break between sections

\section{Education}

\cventry{2015--2020}{Royal Institute of Technology, KTH -- Data Science and Engineering}{}{GPA: 5.0 (out of 5.0)}{}{Master: Statistical Learning and Data Analytics, Bachelor: Engineering Physics. \href{https://kth.diva-portal.org/smash/get/diva2:1431327/FULLTEXT02.pdf}{See thesis.}}

\cventry{2017--2022}{Stockholm University, SU - Economics}{Supplementary Bachelor's Degree}{}{}{}

\cventry{2019--2019}{Swiss Federal Institute of Technology, ETH - Applied Mathematics}{Exchange}{}{}{}
\section{Solo Side Project}
Developing an end-to-end CV system to improve e-commerce navigation by making product images interactive.

\begin{itemize}
  \item Web scraping pipeline to dynamically index all items on a webshop.
  \item Object detection, and classification using YOLOv8, SAM2, and OpenClip to extract all items.
  \item Chromium plugin to display interactive links, improving user navigation.
\end{itemize}

{\footnotesize\textit{Keywords: Computer vision, full-stack ML engineering, fine-tuning, web scraping, YOLOv8, SAM2}}

\hfill{\small{\textit{\hyperref[sec:iris]{Read more…}}}}

% Two blank spaces for break between sections
\section{Skills and Other}

\cvitem{Technology}{Python (PyTorch, TensorFlow, SHAP, YOLO, OpenClip), MS Azure (Data Science Associate Certified), Databricks, Azure DevOps, MongoDB, Qdrant, Docker, 3D Printing, Fusion 360.}
\cvitem{Language}{\textit{Native}: English, Swedish. \textit{Intermediate}: Danish.}
\cvitem{Personal}{In my spare time I enjoy working on projects, such as designing and building furniture.}
\cvitem{Other}{Member of Mensa Sweden.}

\newpage
\section{Detailed Work Experience}

\phantomsection\label{sec:signum}
\cventry{2024--Present}{Data Scientist and Project Manager}{Signum Life Science}{Copenhagen}{}{
    At Signum, I lead development on Kwarts, a next-best-action engine for pharmaceutical sales. I have extended the system with explainable AI features, supported its commercialisation, and ensured scalability on AWS. I also maintain our pharmaceutical price forecasting model for Denmark's biweekly auctions. In addition, I organised a week-long internal hackathon with 40 participants, which fostered innovation and produced several new prototypes.
}{}

\phantomsection\label{sec:valcon}
\cventry{2022--2024}{Principal Data Scientist}{Valcon}{Copenhagen}{}{
    Led data science projects across pharma, energy, IoT, and manufacturing, while mentoring colleagues and shaping internal practices through xAI and Git trainings. I regularly presented to non-technical stakeholders, strengthening my ability to translate complex ideas clearly. As lead data scientist, I designed and deployed a real-time deep neural network with a custom training pipeline for a challenging natural-language, multi-label prediction task. I received the "Valcon People's Choice Award" for exemplary work, leadership, and scientific curiosity.
}{}

\phantomsection\label{sec:valtech}
\cventry{2021--2022}{Data Scientist}{Valtech}{Copenhagen}{}{}{}

\phantomsection\label{sec:raysearch}
\cventry{2019--2021}{Researcher within Data Science}{RaySearch Laboratories}{Stockholm}{}{
    At RaySearch Laboratories, I worked on automating radiotherapy planning using deep learning, probabilistic modelling, and multi-objective optimisation. I built a pipeline that extracted anatomical features from 3D CT scans with autoencoders, identified similar patients, and predicted dose distributions with sparse Gaussian processes, enabling personalised, data-driven treatment planning. I also redesigned lexicographic optimisation algorithms to better reflect clinical decision hierarchies, collaborating closely with physicists and oncologists to ensure clinical relevance.
}{}

\section{Other Detailed Experience}
\phantomsection\label{sec:iris}
\cventry{2025-Present}{Solo Project}{Project iris}{}{}{
    In this personal project I am building an end-to-end computer vision system that makes e-commerce product images interactive. Using YOLOv8 and OpenCLIP, the pipeline detects and embeds items to enable intelligent product linking. It includes a FastAPI backend, MongoDB, Qdrant for vector search, and a custom Chromium plug-in that overlays clickable links on images in real time. The project has been a valuable exercise in fine-tuning deep learning models and full-stack ML engineering.
}{}

\end{document}
