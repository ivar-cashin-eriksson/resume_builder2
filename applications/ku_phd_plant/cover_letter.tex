% %--------------------------
% %	CONFIGURATION
% %--------------------------
\documentclass[11pt,a4paper]{moderncv}
\moderncvstyle{classic}
\moderncvcolor{black}
\definecolor{firstnamecolor}{rgb}{0,0,0}

% Basic packages
\usepackage[utf8]{inputenc}
\usepackage[T1]{fontenc}
\usepackage[scale=0.75]{geometry}

\usepackage{microtype} % For better typography

% Configure font settings
\renewcommand{\rmdefault}{ppl} % Use Palatino font
\renewcommand{\sfdefault}{phv} % Use Helvetica font for sans-serif text
\renewcommand*{\namefont}{\fontsize{26}{18}\mdseries}

% Configure microtype
\microtypesetup{expansion=false}

% Load sensitive information
\input{../../personal_info.tex}

% %--------------------------

\begin{document}
\makecvtitle
Hello Stefan Sommer, I am writing to apply for the PhD position in Data Science and Geometric Models of Morphology. With a Master's degree in data science and applied mathematics, and 6 years of professional experience working with machine learning, I believe I would be a good fit for your open position.

\hspace*{2em}
After several years in industry, I have realised that while building commercial solutions is fun, it does not challange me in the way I would like. Being an extremely creative person, a PhD would allow me to investigate novel ideas that industry does not. I thoroughly enjoy environments of intellectual discussion and rigour, and wish to pursue research again. One strength I bring is the ability to grasp and model complex systems rapidly, which has shown itself in both my work and academic results. I attended The Royal Institute of technology (KTH) in Stockholm, where a lot of weight was placed on understanding the theory behind the methods we were using. In addition to this, I was interested in how this theory could be applied so I took on a bachelor in economics along side my main degree.

\hspace*{2em}
Directly after my studies I continued work at RaySearch Laboratories in Stockholm with whom I had written my Master's thesis. My experience at RaySearch as a Data Science Researcher left a strong, long-lasting impression. I developed probabilistic and ML-based methods for radiotherapy planning of cancer treatment. The work was challenging and rewarding, and involved building custom deep neural networks for computer vision. I particularly enjoyed collaborating with clinicians and all that I learnt in my work there. Though I had to leave Stockholm in 2021 to move to Copenhagen, my motivation to return to more research heavy work has only grown stronger. 

\hspace*{2em}
I am drawn to MorphAI's and DIGIFABA's shared focus on modelling real-world structure, be it anatomical or agricultural, through statistical and machine learning based methods. I value theory-grounded approaches and enjoy building systems that reflect the structure of the world, rather than just fitting data. This is one of the reasons I am now seeking a role where I can contribute to AI research rather than high-level applications.

\hspace*{2em}
Outside of work, I live right by the Botanical Gardens and would love the chance to contribute to a research group embedded in this part of Copenhagen. I fill my free time with designing furniture, woodworking,  coding, 3D-printing, climbing, road biking, ultimate frisbee, and board games, activities that keep my hands and mind sharp. I thoroughly enjoy picking up new skills and am self taught in all my hobbies. In my professional life, I now want to return to a research environment where I can follow ideas from early exploration to experimentation, refinement, and ultimately publication.

\vspace{8 mm}
I hope to hear back from you soon. Regards, 

\vspace{3 mm} 
Ivar Cashin Eriksson.

\end{document}